\documentclass{tufte-handout}

%\geometry{showframe}% for debugging purposes -- displays the margins

\usepackage{amsmath}
\usepackage{amsthm}

% Set up the images/graphics package
\usepackage{graphicx}
\setkeys{Gin}{width=\linewidth,totalheight=\textheight,keepaspectratio}
\graphicspath{{graphics/}}

\title{CPSC 317: Introduction to Computer Networking\thanks{Taught by Prof. Norman Hutchinson}}
\author[The Tufte-LaTeX Developers]{Arpit Kumar}
\date{Fall 2023/24}  % if the \date{} command is left out, the current date will be used


% The following package makes prettier tables.  We're all about the bling!
\usepackage{booktabs}

% The units package provides nice, non-stacked fractions and better spacing
% for units.
\usepackage{units}

% The fancyvrb package lets us customize the formatting of verbatim
% environments.  We use a slightly smaller font.
\usepackage{fancyvrb}
\fvset{fontsize=\normalsize}

% Small sections of multiple columns
\usepackage{multicol}

% Provides paragraphs of dummy text
\usepackage{lipsum}


% code blocks!
\usepackage{listings}
\usepackage{minted}

% These commands are used to pretty-print LaTeX commands
\newcommand{\doccmd}[1]{\texttt{\textbackslash#1}}% command name -- adds backslash automatically
\newcommand{\docopt}[1]{\ensuremath{\langle}\textrm{\textit{#1}}\ensuremath{\rangle}}% optional command argument
\newcommand{\docarg}[1]{\textrm{\textit{#1}}}% (required) command argument
\newenvironment{docspec}{\begin{quote}\noindent}{\end{quote}}% command specification environment
\newcommand{\docenv}[1]{\textsf{#1}}% environment name
\newcommand{\docpkg}[1]{\texttt{#1}}% package name
\newcommand{\doccls}[1]{\texttt{#1}}% document class name
\newcommand{\docclsopt}[1]{\texttt{#1}}% document class option name


\newcommand{\DD}{\mathbb{D}}
\newcommand{\RR}{\mathbb{R}}
\newcommand{\ZZ}{\mathbb{Z}}
\newcommand{\QQ}{\mathbb{Q}}


\newcommand{\DS}{<a_1 \: a_2 \: ... \: a_m \; . \; b_1 \: b_2 \: ...>}
\newcommand{\DSUM}{a_1 \: a_2 \: ... \: a_m + \sum_{j=1}^{k}\frac{b_j}{10^j}}


\newcommand{\fancy}[1]{\mathcal{#1}}

\theoremstyle{definition}
\newtheorem{definition}{Definition}[section]
\newtheorem{lemma}{Lemma}[section]
\newtheorem{corollary}{Corollary}[section]
\newtheorem{warning}{Warning}[section]

\newcommand{\bold}[1]{\textbf{#1}}% bold face

\newcommand{\mgraphics}[2]{
\begin{marginfigure}%
    \includegraphics[width=\linewidth]{#1}
    \caption{#2}
    \label{fig:marginfig}
  \end{marginfigure}}
\begin{document}

\newcommand{\fwgraphics}[2]{
\begin{figure}%
    \includegraphics{#1}
    \caption{#2}
    \label{fig:marginfig}
    \setfloatalignment{b}
  \end{figure}}
\begin{document}

\maketitle% this prints the handout title, author, and date

\begin{abstract}
\noindent These are my class notes, further research and a possibly random collection of facts from this course.
\end{abstract}

%\printclassoptions
\section{1.1 Design of the Internet}
Let's start at the internet's edge and work our way in. 
\begin{itemize}
    \item There are billions of computing devices running network apps on the internet's edge (hosts/end-systems plugged into the internet / hosts called hosts since they host network apps). A few examples of such devices is provided in \textsc{figure 1}. 
    
    \mgraphics{2023-09-22-22-49-31.png}{A few examples of devices you may see connected to the internet}
    

    \item Moving deeper, we find the devices that make the network actually a network, i.e \bold{packet switches:} devices that forward packets (chunks of data). There are two types of packet switches - \bold{routers} and \bold{switches}.

    \item Then we come across the \bold{communication links} that interconnect the routers, hosts and end-systems. 

    \item Finally, each of these components discussed so far are assembled into their own networks administered by some organization/individual. This is why we say that the internet is a network of networks.
    
    \mgraphics{graphics/2023-09-22-23-09-35.png}{A nuts and bolts overview of the internet}
\end{itemize}
  
\fwgraphics{graphics/2023-09-22-23-16-03.png}{A "services" POV of the internet}

\subsection{What are network protocols?}
\begin{itemize}
    \item All communication activity in the interent is governed by protocols
    \item Protocols define the format, order of messages sent and recieved among network entities, and actions taken on message transmission and receipt
\end{itemize}

All communication over the internet requires a \bold{protocol}, \bold{source}, \bold{destination}, \bold{communication medium}, and of course, a \bold{message}.

\section{1.2 The network edge}
All about the big picture here. Details further along in the course.
\fwgraphics{graphics/2023-09-23-14-58-31.png}{Access networks connect end-systems to the internet}

\subsection*{Types of access networks}
\begin{itemize}
    \item \bold{Cable-based access network:} \bold{asymmetric} (downloads faster than uploads), homes share access network to cable headend, modem rate limits your speeds (you get what you pay for)
    \mgraphics{graphics/2023-09-23-15-01-43.png}{There only so many channels, so some users need to share the same channel (more in chapter 6)}
    \item \bold{Digital subscriber line (DSL):} use exisiting telephone line (voice, data transmitted at different frequencies) to central office DSLAM (access modem), also \bold{asymmetric}, speeds depend on distance to central office, can't do DSL if distance over ~3 miles
    \item \bold{Home networks:} \fwgraphics{graphics/2023-09-23-15-08-53.png}{Home networks setup}
    \item \bold{Wireless networks}
    Shared wirless access networks connect end-systems to router via a base station aka "access point" \\
    
    \begin{itemize}
        \item \bold{Wireless Local Area Networks (WLANs):} WiFi, operate within around $\sim$ 100ft 
        \item \bold{ Wide-area cellular access networks:} 3G/4G/5G, provided by mobile, cellular network operator
    \end{itemize}

    \item \bold{Enterprise network:} home network on steriods, mix of wired, wireless technologies, connecting a mix of switches and routers
    \item \bold{Data center network:} connect hundreds to thousands of servers together over high bandwidth links (10s to 100s Gbps)
\end{itemize}

\subsection{Links: physical media}
\begin{itemize}
    \item \bold{Bit}: propogates between reciever/transmitter pair
    \item \bold{Physical link}: what lies between transmitter and reciever
    \item \bold{Guided media}: signals propogate in solid media (copper, fiber, coax)
    \item \bold{Unguided media}: signals propogate freely
    \item \bold{Twisted pair (TP)}: two insulated copper wires, now refers to ethernet, ADSL (Asymmetric Digital Subscriber Line)
    \item \bold{Coaxial cable}: two concentric copper conductors, broadband connection (100's Mbps over multiple channels)
    \item \bold{Fiber optic cable}: high speed operation (10's - 100's Gbps), low error rate \mgraphics{graphics/2023-09-23-15-34-09.png}{Fiber optics cable}
    \item \bold{Wireless radio}: signals carried in various bands in the EM spectrum, broadcast (anyone can recieve - eavsdropping, interference concerns), signal fades over distance, obstruction by objects, affected by noise 
        \begin{itemize}
            \item Wireless LAN (WiFi)
            \item Wide-area (4G cellular)
            \item Bluetooth - short distances with limited rates
            \item Terrestrial Microwave
            \item Satellite
        \end{itemize}
\end{itemize}


\section*{1.3 Internet core}
Essentially a mesh of routers connected by some connection link. The internet's core operation is based on a principle known as \bold{packet-switching}: hosts break application layer messages into packets and the network forwards packets from one router to the next across links on path from source to destination.

Two key network core function:
\begin{itemize}
    \item Forwarding (aka switching): a \bold{local} action that maps packets at input link to appropriate output link
    \item Routing: how does the router know where to forward packets to? Routing is a \bold{global} action to determine the path between a source and destination
\end{itemize}

\emph{Store and forward:} entire packet must arrive at a router before it can be forwarded

\emph{Queue:} packets waiting at the router when the arival rate is faster than the transmission rate 
\begin{itemize}
    \item packets can be dropped if memory (buffer) in router fills up (packet loss)
\end{itemize}

\subsection*{Alternate to packet switching: circuit switching}
\begin{itemize}
    \item Dedicated resources, no sharing 
    \item Circuit segment remains idle if not used by call (no sharing)
\end{itemize}

\mgraphics{graphics/2023-09-24-13-19-25.png}{insane}

\bold{Multiplexing} describes multiple input streams sharing the same medium
\fwgraphics{graphics/2023-09-24-13-07-03.png}{Circuit switching: FDM and TDM}

\bold{Pakcet switching} is a more viable option than \bold{circuit switching} due to \bold{statistical multiplexing}, i.e not all users are active at the same time and hence the probability that packets are going to begin queueing up is low. 

\fwgraphics{graphics/2023-09-24-13-25-29.png}{Is packet switching a 
"slam dunk winner"}

\section{1.5 Layering, Encapsulation}
Why have a layered architecture? Because \bold{explicit structure allows identification, relationship of system's pieces} and \bold{modularization eases maintainence, updating of system}

\fwgraphics{graphics/2023-09-30-01-31-06.png}{(a) 5-layer internet protocol stack (b) 7-layer ISO OSI reference model}

Each layer has its own set of protocols to choose from.

\fwgraphics{graphics/2023-09-24-13-39-47.png}{Encapsulation: adding header from packet of the layer above}

Each layer takes data from above
\begin{itemize}
    \item adds header information to create a new data unit
    \item passes new data unit into the layer below
    \item recieving system passes these data units up the stack
    \item recieveing system reads, parses then unpacks each data unit and sends it to the layer above
\end{itemize}

\mgraphics{graphics/2023-09-24-13-41-42.png}{An example}

Note that routers and switches don't operate on higher levels of the stack since they only deal with routing and forwarding data packets

Okay, now let's describe what each layer in this stack is responsible for. 
@TODO

\section{Application Layer}
@TODO   


\section{Transport Layer}
Provides logical communication between application processes between different hosts. Handles breaking down messages into \textbf{segments} and reassembly at the reciever end. Also provides \textbf{multiplexing} of communication over the network. It enhances the services of the network layer by providing services such as congestion control, reliable delivery, etc.

\subsection{Transport layer vs Network layer}
Network layer terminates at the interface while the transport layer terminates once the message has delivered to the application process (a particular port on the host).

\mgraphics{graphics/2023-10-27-15-30-55.png}{The big picture of transport protocols}

Two main transport layer protocols available to internet applications include \textbf{UDP} and \textbf{TCP}. UDP is unreliable. TCP is reliable. This distinction is important and taken into consideration by application layer protocols to decide which protocol to use.

\section{UDP}
\fwgraphics{graphics/2023-10-27-15-38-26.png}{UDP segment format}

Notice the checksum field - its purpose is to detect errors that may have occurred during transmission.\footnote{Checksums appear at the transport layer, network layer and link layer. They serve a different purpose at each of these layers. The same algorithm is used to compute the checksum at the transport and network layer.}

\subsection{Checksum Algorithm}
\begin{itemize}
    \item Treat data as a sequence of 16-bit integers
    \item Take the 1's complement sum of all these of 16 bit integers
    \item Checksum is the 1's complement of the computed value
    \item To verify, add this checksum to the sequence of integers under consideration and repeat the steps above - valid if you get all 0s
\end{itemize}

\mgraphics{graphics/2023-10-27-15-51-14.png}{Compute checksum}
\mgraphics{graphics/2023-10-27-15-51-35.png}{Verify checksum}

\section{State Machines and Reliability}
\subsection{Again, what is a protcol?}

A \textbf{protocol} defines the \underline{order and format of messages} between two communicating entities, as well as the \underline{actions taken} on the transmission and/or receipt of a message or other event.


\subsection{Alternating Bit Protocol}
Alternating bit protocol is a reliable transport layer protocol that gurantees delivery and ensures no duplication.

Services provided by the ABP are summarized below:
\begin{itemize}
    \item Send only one segment at a time
    \item Identify when sending is allowable action
    \item Identify when re-sending is required
    \item Enumerate events and actions for both sender and receiver
\end{itemize}

\fwgraphics{graphics/2023-10-27-16-19-57.png}{The alternating bit protocol (stop and wait protocol)}

\mgraphics{graphics/2023-10-27-16-20-13.png}{Can be simplified to not have any explicit NAKs}

\section{Lost Segments and Timeouts}
Now it may be that data is lost in transmission. We can further simplify our protocol by treating corrupt data as lost data, i.e instead of sending a NAK, we just don't reply. The sender keeps track of the time elapsed since the message was sent, and if the an ACK is not received in time, the message is sent again. 

\fwgraphics{graphics/2023-10-27-19-36-13.png}{Handling lost data}
\mgraphics{graphics/2023-10-27-19-37-26.png}{Timeout too soon}

The length of the timeout is crucial. It is clear that our timeout must be longer than the RTT of our connection. Otherwise, unescessary retransmissions will be sent. To decide on the timeout value, we need an estimate of the RTT values. TCP maintains a RTT measurement of one of the transmitted and yet to be acknowledged segments (ignoring retransmissions). This means that a new sampleRTT value is obtained once every RTT. This new value is used to update the estimated RTT by the following formula
$$EstimatedRTT = (1 - \alpha) \cdot EstimatedRTT + \alpha \cdot SampleRTT$$
Recent RTT values should be weighted more heavily. Thus, a suggested $\alpha$ value is $0.125$.

Similarly, a measure of the deviation in the RTT is kept and updated as follows
$$DevRTT = (1 - \beta) \cdot DevRTT + \beta \cdot |SampleRTT - EstimatedRTT|$$

A recommneded value for $\beta = 0.25$. It is desirable to set the timeout equal to the EstimatedRTT plus some margin. The margin should be large when there is a lot of fluctuation in the SampleRTT values; it should be small when there is little fluctuation. The value of
DevRTT should thus come into play here. All of these considerations are taken into account in TCP's method for determining the retransmission timeout interval:
$$TimeoutInterval = EstimatedRTT + 4 \cdot DevRTT $$

\mgraphics{graphics/2023-10-27-20-09-54.png}{RTT samples and RTT estimates: our estimation formula smoothens out large fluctuations in RTT}

\section{Windowing Protocols}

One major flaw with the ABP we discussed is the fact that it waits for an ACK before transmitting the next segment. This means that network resources are severly under-utilized.

\mgraphics{graphics/2023-10-27-20-47-07.png}{ABP underutilizes network bandwidth}
\marginnote{Utilization refers to the freaction of time that a sender is busy sening data}

Well, to solve this issue, the sender must be able to send mutliple segments without waiting for their ACK. This is known as pipelining. We will look at two pipeline reliable protocol strategies - \textbf{Go-Back-N} and \textbf{Selective Repeat}.

\subsection{Go-Back-N}
@TODO \footnote{Think about message reordering}

\subsection{Selective Repeat}
@TODO


\subsection{Flow and Congestion Control}
@TODO


\section{Alternate Transport Protocols}
@TODO

\section{Network Layer}
\subsection{History and Autonomous Systems}






\end{document}
