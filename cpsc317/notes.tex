\documentclass{tufte-handout}

%\geometry{showframe}% for debugging purposes -- displays the margins

\usepackage{amsmath}
\usepackage{amsthm}

% Set up the images/graphics package
\usepackage{graphicx}
\setkeys{Gin}{width=\linewidth,totalheight=\textheight,keepaspectratio}
\graphicspath{{graphics/}}

\title{CPSC 317: Introduction to Computer Networking\thanks{Taught by Prof. Norman Hutchinson}}
\author[The Tufte-LaTeX Developers]{Arpit Kumar}
\date{Fall 2023/24}  % if the \date{} command is left out, the current date will be used


% The following package makes prettier tables.  We're all about the bling!
\usepackage{booktabs}

% The units package provides nice, non-stacked fractions and better spacing
% for units.
\usepackage{units}

% The fancyvrb package lets us customize the formatting of verbatim
% environments.  We use a slightly smaller font.
\usepackage{fancyvrb}
\fvset{fontsize=\normalsize}

% Small sections of multiple columns
\usepackage{multicol}

% Provides paragraphs of dummy text
\usepackage{lipsum}


% code blocks!
\usepackage{listings}
\usepackage{minted}

% These commands are used to pretty-print LaTeX commands
\newcommand{\doccmd}[1]{\texttt{\textbackslash#1}}% command name -- adds backslash automatically
\newcommand{\docopt}[1]{\ensuremath{\langle}\textrm{\textit{#1}}\ensuremath{\rangle}}% optional command argument
\newcommand{\docarg}[1]{\textrm{\textit{#1}}}% (required) command argument
\newenvironment{docspec}{\begin{quote}\noindent}{\end{quote}}% command specification environment
\newcommand{\docenv}[1]{\textsf{#1}}% environment name
\newcommand{\docpkg}[1]{\texttt{#1}}% package name
\newcommand{\doccls}[1]{\texttt{#1}}% document class name
\newcommand{\docclsopt}[1]{\texttt{#1}}% document class option name


\newcommand{\DD}{\mathbb{D}}
\newcommand{\RR}{\mathbb{R}}
\newcommand{\ZZ}{\mathbb{Z}}
\newcommand{\QQ}{\mathbb{Q}}


\newcommand{\DS}{<a_1 \: a_2 \: ... \: a_m \; . \; b_1 \: b_2 \: ...>}
\newcommand{\DSUM}{a_1 \: a_2 \: ... \: a_m + \sum_{j=1}^{k}\frac{b_j}{10^j}}


\newcommand{\fancy}[1]{\mathcal{#1}}

\theoremstyle{definition}
\newtheorem{definition}{Definition}[section]
\newtheorem{lemma}{Lemma}[section]
\newtheorem{corollary}{Corollary}[section]
\newtheorem{warning}{Warning}[section]

\newcommand{\bold}[1]{\textbf{#1}}% bold face

\newcommand{\mgraphics}[2]{
\begin{marginfigure}%
    \includegraphics[width=\linewidth]{#1}
    \caption{#2}
    % \label{fig:marginfig}
  \end{marginfigure}}
\begin{document}

\newcommand{\fwgraphics}[2]{
\begin{figure}%
    \includegraphics{#1}
    \caption{#2}
    % \label{fig:marginfig}
    \setfloatalignment{b}
  \end{figure}}
\begin{document}

\maketitle% this prints the handout title, author, and date

\begin{abstract}
\noindent These are my class notes, further research and a possibly random collection of facts from this course.
\end{abstract}

%\printclassoptions
\section{1.1 Design of the Internet}
Let's start at the internet's edge and work our way in. 
\begin{itemize}
    \item There are billions of computing devices running network apps on the internet's edge (hosts/end-systems plugged into the internet / hosts called hosts since they host network apps). A few examples of such devices is provided in \textsc{figure 1}. 
    
    \mgraphics{2023-09-22-22-49-31.png}{A few examples of devices you may see connected to the internet}
    

    \item Moving deeper, we find the devices that make the network actually a network, i.e \bold{packet switches:} devices that forward packets (chunks of data). There are two types of packet switches - \bold{routers} and \bold{switches}.

    \item Then we come across the \bold{communication links} that interconnect the routers, hosts and end-systems. 

    \item Finally, each of these components discussed so far are assembled into their own networks administered by some organization/individual. This is why we say that the internet is a network of networks.
    
    \mgraphics{graphics/2023-09-22-23-09-35.png}{A nuts and bolts overview of the internet}
\end{itemize}
  
\fwgraphics{graphics/2023-09-22-23-16-03.png}{A "services" POV of the internet}

\subsection{What are network protocols?}
\begin{itemize}
    \item All communication activity in the interent is governed by protocols
    \item Protocols define the format, order of messages sent and recieved among network entities, and actions taken on message transmission and receipt
\end{itemize}

All communication over the internet requires a \bold{protocol}, \bold{source}, \bold{destination}, \bold{communication medium}, and of course, a \bold{message}.

\section{1.2 The network edge}
All about the big picture here. Details further along in the course.
\fwgraphics{graphics/2023-09-23-14-58-31.png}{Access networks connect end-systems to the internet}

\subsection*{Types of access networks}
\begin{itemize}
    \item \bold{Cable-based access network:} \bold{asymmetric} (downloads faster than uploads), homes share access network to cable headend, modem rate limits your speeds (you get what you pay for)
    \mgraphics{graphics/2023-09-23-15-01-43.png}{There only so many channels, so some users need to share the same channel (more in chapter 6)}
    \item \bold{Digital subscriber line (DSL):} use exisiting telephone line (voice, data transmitted at different frequencies) to central office DSLAM (access modem), also \bold{asymmetric}, speeds depend on distance to central office, can't do DSL if distance over ~3 miles
    \item \bold{Home networks:} \fwgraphics{graphics/2023-09-23-15-08-53.png}{Home networks setup}
    \item \bold{Wireless networks}
    Shared wirless access networks connect end-systems to router via a base station aka "access point" \\
    
    \begin{itemize}
        \item \bold{Wireless Local Area Networks (WLANs):} WiFi, operate within around $\sim$ 100ft 
        \item \bold{ Wide-area cellular access networks:} 3G/4G/5G, provided by mobile, cellular network operator
    \end{itemize}

    \item \bold{Enterprise network:} home network on steriods, mix of wired, wireless technologies, connecting a mix of switches and routers
    \item \bold{Data center network:} connect hundreds to thousands of servers together over high bandwidth links (10s to 100s Gbps)
\end{itemize}

\subsection{Links: physical media}
\begin{itemize}
    \item \bold{Bit}: propogates between reciever/transmitter pair
    \item \bold{Physical link}: what lies between transmitter and reciever
    \item \bold{Guided media}: signals propogate in solid media (copper, fiber, coax)
    \item \bold{Unguided media}: signals propogate freely
    \item \bold{Twisted pair (TP)}: two insulated copper wires, now refers to ethernet, ADSL (Asymmetric Digital Subscriber Line)
    \item \bold{Coaxial cable}: two concentric copper conductors, broadband connection (100's Mbps over multiple channels)
    \item \bold{Fiber optic cable}: high speed operation (10's - 100's Gbps), low error rate \mgraphics{graphics/2023-09-23-15-34-09.png}{Fiber optics cable}
    \item \bold{Wireless radio}: signals carried in various bands in the EM spectrum, broadcast (anyone can recieve - eavsdropping, interference concerns), signal fades over distance, obstruction by objects, affected by noise 
        \begin{itemize}
            \item Wireless LAN (WiFi)
            \item Wide-area (4G cellular)
            \item Bluetooth - short distances with limited rates
            \item Terrestrial Microwave
            \item Satellite
        \end{itemize}
\end{itemize}


\section*{1.3 Internet core}
Essentially a mesh of routers connected by some connection link. The internet's core operation is based on a principle known as \bold{packet-switching}: hosts break application layer messages into packets and the network forwards packets from one router to the next across links on path from source to destination.

Two key network core function:
\begin{itemize}
    \item Forwarding (aka switching): a \bold{local} action that maps packets at input link to appropriate output link
    \item Routing: how does the router know where to forward packets to? Routing is a \bold{global} action to determine the path between a source and destination
\end{itemize}

\emph{Store and forward:} entire packet must arrive at a router before it can be forwarded

\emph{Queue:} packets waiting at the router when the arival rate is faster than the transmission rate 
\begin{itemize}
    \item packets can be dropped if memory (buffer) in router fills up (packet loss)
\end{itemize}

\subsection*{Alternate to packet switching: circuit switching}
\begin{itemize}
    \item Dedicated resources, no sharing 
    \item Circuit segment remains idle if not used by call (no sharing)
\end{itemize}

\mgraphics{graphics/2023-09-24-13-19-25.png}{insane}

\bold{Multiplexing} describes multiple input streams sharing the same medium
\fwgraphics{graphics/2023-09-24-13-07-03.png}{Circuit switching: FDM and TDM}

\bold{Pakcet switching} is a more viable option than \bold{circuit switching} due to \bold{statistical multiplexing}, i.e not all users are active at the same time and hence the probability that packets are going to begin queueing up is low. 

\fwgraphics{graphics/2023-09-24-13-25-29.png}{Is packet switching a 
"slam dunk winner"}

\section{1.5 Layering, Encapsulation}
Why have a layered architecture? Because \bold{explicit structure allows identification, relationship of system's pieces} and \bold{modularization eases maintainence, updating of system}

\fwgraphics{graphics/2023-09-30-01-31-06.png}{(a) 5-layer internet protocol stack (b) 7-layer ISO OSI reference model}

Each layer has its own set of protocols to choose from.

\fwgraphics{graphics/2023-09-24-13-39-47.png}{Encapsulation: adding header from packet of the layer above}

Each layer takes data from above
\begin{itemize}
    \item adds header information to create a new data unit
    \item passes new data unit into the layer below
    \item recieving system passes these data units up the stack
    \item recieveing system reads, parses then unpacks each data unit and sends it to the layer above
\end{itemize}

\mgraphics{graphics/2023-09-24-13-41-42.png}{An example}

Note that routers and switches don't operate on higher levels of the stack since they only deal with routing and forwarding data packets

Okay, now let's describe what each layer in this stack is responsible for. 
@TODO

\section{Application Layer}
@TODO   


\section{Transport Layer}
Provides logical communication between application processes between different hosts. Handles breaking down messages into \textbf{segments} and reassembly at the reciever end. Also provides \textbf{multiplexing} of communication over the network. It enhances the services of the network layer by providing services such as congestion control, reliable delivery, etc.

\subsection{Transport layer vs Network layer}
Network layer terminates at the interface while the transport layer terminates once the message has delivered to the application process (a particular port on the host).

\mgraphics{graphics/2023-10-27-15-30-55.png}{The big picture of transport protocols}

Two main transport layer protocols available to internet applications include \textbf{UDP} and \textbf{TCP}. UDP is unreliable. TCP is reliable. This distinction is important and taken into consideration by application layer protocols to decide which protocol to use.

\section{UDP}
\fwgraphics{graphics/2023-10-27-15-38-26.png}{UDP segment format}

Notice the checksum field - its purpose is to detect errors that may have occurred during transmission.\footnote{Checksums appear at the transport layer, network layer and link layer. They serve a different purpose at each of these layers. The same algorithm is used to compute the checksum at the transport and network layer.}

\subsection{Checksum Algorithm}
\begin{itemize}
    \item Treat data as a sequence of 16-bit integers
    \item Take the 1's complement sum of all these of 16 bit integers
    \item Checksum is the 1's complement of the computed value
    \item To verify, add this checksum to the sequence of integers under consideration and repeat the steps above - valid if you get all 0s
\end{itemize}

\mgraphics{graphics/2023-10-27-15-51-14.png}{Compute checksum}
\mgraphics{graphics/2023-10-27-15-51-35.png}{Verify checksum}

\section{Transport Layer Multiplexing and De-multiplexing}
A host recieves IP datagrams (more on this later) - each datagram contains the source and desitnation IP addresses and one transport layer segment. Each segment contains the source and destination port numbers. The host uses IP addresses and port numbers to direct the segment to the appropriate socket.
Since UDP only de-multiplexes on the basis of destination port number, it sends all datagrams from all origins to the same socket. TCP de-mutliplexes on the basis of the tuple (source IP, destination IP, source port, destination port). This is because TCP is connection oriented.


\section{State Machines and Reliability}
\subsection{Again, what is a protcol?}

A \textbf{protocol} defines the \underline{order and format of messages} between two communicating entities, as well as the \underline{actions taken} on the transmission and/or receipt of a message or other event.


\subsection{Alternating Bit Protocol}
Alternating bit protocol is a reliable transport layer protocol that gurantees delivery and ensures no duplication.

Services provided by the ABP are summarized below:
\begin{itemize}
    \item Send only one segment at a time
    \item Identify when sending is allowable action
    \item Identify when re-sending is required
    \item Enumerate events and actions for both sender and receiver
\end{itemize}

\fwgraphics{graphics/2023-10-27-16-19-57.png}{The alternating bit protocol (stop and wait protocol)}

\mgraphics{graphics/2023-10-27-16-20-13.png}{Can be simplified to not have any explicit NAKs}

\section{Lost Segments and Timeouts}
Now it may be that data is lost in transmission. We can further simplify our protocol by treating corrupt data as lost data, i.e instead of sending a NAK, we just don't reply. The sender keeps track of the time elapsed since the message was sent, and if the an ACK is not received in time, the message is sent again. 

\fwgraphics{graphics/2023-10-27-19-36-13.png}{Handling lost data}
\mgraphics{graphics/2023-10-27-19-37-26.png}{Timeout too soon}

The length of the timeout is crucial. It is clear that our timeout must be longer than the RTT of our connection. Otherwise, unescessary retransmissions will be sent. To decide on the timeout value, we need an estimate of the RTT values. TCP maintains a RTT measurement of one of the transmitted and yet to be acknowledged segments (ignoring retransmissions). This means that a new sampleRTT value is obtained once every RTT. This new value is used to update the estimated RTT by the following formula
$$EstimatedRTT = (1 - \alpha) \cdot EstimatedRTT + \alpha \cdot SampleRTT$$
Recent RTT values should be weighted more heavily. Thus, a suggested $\alpha$ value is $0.125$.

Similarly, a measure of the deviation in the RTT is kept and updated as follows
$$DevRTT = (1 - \beta) \cdot DevRTT + \beta \cdot |SampleRTT - EstimatedRTT|$$

A recommneded value for $\beta = 0.25$. It is desirable to set the timeout equal to the EstimatedRTT plus some margin. The margin should be large when there is a lot of fluctuation in the SampleRTT values; it should be small when there is little fluctuation. The value of
DevRTT should thus come into play here. All of these considerations are taken into account in TCP's method for determining the retransmission timeout interval:
$$TimeoutInterval = EstimatedRTT + 4 \cdot DevRTT $$

\mgraphics{graphics/2023-10-27-20-09-54.png}{RTT samples and RTT estimates: our estimation formula smoothens out large fluctuations in RTT}

\section{Windowing Protocols}

One major flaw with the ABP we discussed is the fact that it waits for an ACK before transmitting the next segment. This means that network resources are severly under-utilized.

\mgraphics{graphics/2023-10-27-20-47-07.png}{ABP underutilizes network bandwidth}
\marginnote{Utilization refers to the freaction of time that a sender is busy sening data}

Well, to solve this issue, the sender must be able to send mutliple segments without waiting for their ACK. This is known as pipelining. We will look at two pipeline reliable protocol strategies - \textbf{Go-Back-N} and \textbf{Selective Repeat}.

\subsection{Go-Back-N}
@TODO \footnote{Think about message reordering}

\subsection{Selective Repeat}
@TODO


\subsection{Flow and Congestion Control}
@TODO


\section{Alternate Transport Protocols}
@TODO

\section{Network Layer}
\mgraphics{graphics/2023-12-06-12-15-13.png}{The network service model}
The network layer is implemented in each and every device connected to the internet and is really the glue that holds the internet together. It serves two functions:
\begin{itemize}
    \item \textbf{Forwarding:} move packets from a router's input link to appropriate output line
    \item \textbf{Routing:} determine the route taken by packets from source to destination
\end{itemize}

\mgraphics{graphics/2023-11-11-10-05-21.png}{The network layer, notice how the IP protocol has nothing to do with the routing algorithm - that is a control plane function}


The network layer has two sub-layers or planes:

\begin{itemize}
    \item \textbf{Data Plane}: local, per-router function that determines how datagram arriving at router is forwarded to output port
    \item \textbf{Control Plane:} determines how datagram is routed across routers
\end{itemize}
Routers are the principal network devices in the network core, with a fairly simple job: examine the headaer fields in all IP datagrams passing through it and move them from input to output ports to enable transfer of the datagram along an end-to-end path.


\subsection{History and Autonomous Systems}
The internet is not really a network of end-systems - its a network of ISPs that have autonomous control over how they operate their share of the internet (i.e choosing what routing algorithms to use, etc). And so routers are organized into \textit{autonomous systems}, with each AS consisting of routers that under the same administrative control. Each AS is identified by its globally unique autonomous system number. Routers within the same AS all run the same routing algorithm and each have information about one another.

Tier-1, Tier-2, Tier-3 networks. 



\subsection{IP Addresses and Forwarding}
\mgraphics{graphics/2023-11-11-10-03-31.png}{IPV6 Datagram}
\fwgraphics{graphics/2023-11-11-10-07-55.png}{IPV4 Datagram, 20 bytes of IP overhead, 20 bytes of TCP overhead + application overhead}

An IP address identifies \textbf{an interface} on the host, and not just the host. This is because a host connected to the internet can have multiple interfaces (WiFi, ethernet, etc).

Device interfaces that can physically reach each other without passing through an intervening router are in the same \textbf{subnet}. IP addresses are structured to identify the subnet a host belongs in, then identifies the host (with the lower order bits) in that subnet.
\mgraphics{graphics/2023-11-11-10-12-57.png}{Subnets}

\marginnote{An octet is RFC-speak for a byte - anciently different computers had different sized bytes, hence an octet is more precise.}

\subsection{A bit of IP history (pun intended)}

\mgraphics{graphics/2023-12-06-11-44-10.png}{The IP protocol is what ties the internet together - is known as the waist of the internet}

In the original IP addresses, first  8 bits represented the network and the remaining 24 bits identified the host. For example, $17.0.0.1$ was the first host on the apple network. Note that first (i.e host address all 0s) and the last addresses (host address all 1s) is never assigned and used for different purposes. The last address is usually used to broadcast messages to all hosts in a subent, while the first address represent a host connected to a network but not yet assigned an IP.\footnote{/31 address are exempt from this convention}

Because we \textbf{underestimated} the number of networks we would need (no one expected that one day, everyone would own multiple IPs and that there would so many networks), original IP addressing (described above) quickly became obsolete - 8 bits were not enough to represent all networks. And so, a new scheme was introduced - \textbf{Class Based Addressing}

\mgraphics{graphics/2023-11-11-10-25-26.png}{Class Based Addressing}

But this wasn't enough, so the scheme was complicated further still, and gave rise to \textbf{CIDR}. The subnest mask $/24$ or in general $/{x}$ identifys how many top order bits are used in the subnet part of an IP. IP addresses in format $a.b.c.d/x$ are known as Classless InterDomain Routing (or CIDR) addresses.\footnote{CIDR addresses are in dominant use now}

\mgraphics{graphics/2023-11-11-10-32-53.png}{Alternate format of representing CIDR addresses}

But this still wasn't enough, and so introducing IP$V6$ addresses.
@TODO

\subsection{IP Address Ranges}
Now, we answer two questions:
\begin{itemize}
    \item How does a host within a network get assigned an IP address?
    \item How does a network itself get assigned an IP address?
\end{itemize}

Let's tackle the first question. A host is either hardcoded an IP by the sys-admin (for eg. in \etc\rc.config for UNIX machines) or through the Dynamic Host Configuration Protocol (DHCP). A DHCP servers typically serves one router and all the subnets attached to that router. 

\mgraphics{graphics/2023-11-11-10-52-40.png}{A simple DHCP interaction}

Ok, now for the second question, how does a network get its subnet address? Well, it gets allocated a portion of its providers ISP's address space.
\mgraphics{graphics/2023-11-11-10-55-44.png}{ISP dividing its address space for different networks}

Notice that the rest of the internet really only needs the ISP's IP to access each of the 8 networks above. Route/Address Aggregation!

\fwgraphics{graphics/2023-11-11-10-57-17.png}{Route or Address Aggregation}

Its never this organized in the real world. Organizations might jump ship and choose a different ISP, while still keeping their original IP address ranges. This is why the internet forwards packets based on the \textbf{longest prefix match}.
\mgraphics{graphics/2023-11-11-11-00-38.png}{Longest Prefix Matching}
Ok, so how do ISPs get their IP address? Well, its a complicated process of going through registrars and so forth.

\mgraphics{graphics/2023-11-11-11-07-35.png}{Regional Internet Registeries}


\subsection{Routing}
IGP vs EGP
This is the problem that falls into the control-plane of the network layer. Routing algorithms can be clasified into the following.
% \newpage
\fwgraphics{graphics/2023-11-11-12-12-07.png}{Classification of routing algorithms}

\subsection{Link State Routing}
Each router has the link information for every other router. Then just uses Dijikstra's algorithm to determine shortest path from it to every other router.

\subsection{Distance Vector Routing}
From time to time, each node sends its own distance distance vector estimates to its neighbours. 


\subsection{BGP}
Along with IP, makes up the two most important algorithms in all of computer networking. Its the "glue" that holds internet together. BGP allows a network to advertise its existence to the rest of the internet. It provides an AS
\begin{itemize}
    \item a means to obtain network reachability information from neighbouring ASes (eBGP)
    \item determine routes to networks outside AS based on reachability information and policy
    \item propogate reachability information to all internal AS routers (iBGP)
    \item advertise to neighbouring networks destination reachability information, again based on policy
\end{itemize}

\mgraphics{graphics/2023-12-06-16-46-42.png}{eBGP, iBGP}

\section{Network Address Translation (NAT)}
We are out of IPv4 addresses. For a few years now. Yet, instead of switching over to IPv6, we play tricks and pretend we have more addresses than we actually do. We can do this becase:
\begin{itemize}
    \item Do we really need globally unique IP addresses?
    \item Other hosts \textit{rarely} initiate a direct connection with you - most connections target servers "out there" in the internet 
    \item So, maybe several hosts could share a single IP address 
\end{itemize}

The idea is simple. Hosts on a private network each get assigned an internal IP non-routable address. These hosts are then serviced by a NAT router, whose public IP address is used to reference every host in this network. The NAT router takes care of the mappings by maintaining a forwarding table.

\fwgraphics{graphics/2023-12-13-07-15-48.png}{NAT forwarding table}

But...what if a connection comes from the outside\footnote{known as the NAT traversal problem}, or if the remote host too is behind a NAT (its reply will have the NAT address as the sender) - each of these cases would not match any entry in the forwarding table. We could resolve this issue by manually adding a fixed rule - say if a connection comes to address X at port Y, forward to internal host Z. Another approach is to employ UPnP (Universal Plug and Play). This is a protocol that allows host to communicate to the NAT router what traffic to forward to them\footnote{Nasser Hussein's video on UPnP shows you can example of doing so}.

NAT has been controversial - a network-level device mucking around with IP addresses and port numbers disobeys the layered responsibilities of networking, and a purist might say, let's just bite the bullet and switch to IPv6.

\section{Link Layer}
Hosts and routers are nodes in the graph that is the internet. The edges connecting these nodes are the "links" we now talk about. These can be wired or wireless or connections over a LAN (switches). Link layer packets are known as frames and encapsulate a datagram. The primary resposibility of the link layer is to transfer a packet from one host (more precisely, interface) to another over a phyical medium. To do so, they also need to manage shared access to a medium - media access control.

There are two broad categories of links:
\begin{itemize}
    \item Point-to-point: direct link between two interfaces 
    \item  Broadcast links: shared between multiple links 
\end{itemize}

\mgraphics{graphics/2023-12-13-09-20-41.png}{Link layer frame}

The link layer makes use of MAC addresses to identify interfaces. 
\subsection{MAC vs IP}
\begin{itemize}
    \item IP is 32-bit or 128-bit, MAC is 48-bit 
    \item IP addresses must be unique globally, while MAC must be unique in the network
    \item MAC addresses are burned into the ROM of the adapater - doesn't change while IP addresses depends on which subnet the host is attached to 
    \item Notice that MAC addresses share no relationship with another - unlike IP where addresses on the same network share their prefix, this means we cannot use any from longest prefix matching when forwarding frames using MAC addresses 
\end{itemize}


\subsection{Error detection and correction}
\mgraphics{graphics/2023-12-13-09-21-43.png}{EDC pipeline}

\begin{itemize}
    \item Single bit parity check: add a parity bit to maintain even parity
    \item 2-d partiy check: add a parity bit to each row, column and an overall parity bit, can correct single bit errors and detect upto 3-bit errors
    \item Checksums, such as the internet checksum
    \item Cyclic redundancy check (CRC): used to detect burst errors, which are rather common in practice... 
        \begin{itemize}
            \item parameterized by constants G, r - where G is the generator, an arbitrary bit pattern and r+1 is the length of the generator 
            \item Some generators are better than others - given in the CRC standard 
            \item When the sender wants to send D, they choose r CRC bits, R, such that <D,R> is exactly divisible by G modulo 2 
            \item The receiver knows G, divides <D,R> by G, if non-zero remainder an error has been detected
            \item The computation of R and derivation of G can be implmented quite quickly in hardware 
        \end{itemize}
\end{itemize}

\subsection{Access control and ARP}
Half-duplex links are not uncommon at the link layer. If a link is half-duplex, how does a potential sender know when the link is busy? They listen. Listening is also used to detect "collision" (when two or more senders try to use the link at once). If a collision is detected or the link is busy, the senders wait a random amount of time before trying again. This kind of a protocol for access control is referred to as CSMA (Carrier sense multiple access), and it matches our human conventions for conversation. CSMA/CD adds collision detection and aborts sending if a collision is detected (again matching the human convention of conversation). The wait time before trying again is usually "binary exponential backoff" - where we choose a random number in the range $1-2^n$ where $n$ is the number of attempts to send that have failed due the link being busy or a collision.

Another approach to access control is "turn-based" access control where the senders take turn using the link and pass their turn if they don't need to use the link at the time. This type of a system could either be implemented in a centralized or decentralized fashion. In the centralized control version, the single control node polls every sender to see if they want a turn - there is a way for senders to signal that they would want to use the link. This is used in WiFi connections. In the decentralized version, nodes pass around a token and only the sender with the token may use the link. This introduces extra complexity - what happens if the node with the token breaks. Or what if a new token is generated even though the old one is still valid?

In general, CSMA/CD will use the network resources efficiently when there is just one host sending data - there will be no collisions and the host can send data at full speed. In the case with many senders, turn-based access control is a lot more efficient - there will rarely be a sender who wouldn't have data to send on their turn.

Now let's talk about switches - switches provide a full duplex connection between nodes in a LAN and also provide broadcast functionality. A switch contains a forwarding table with 3 columns - MAC address, interface, time. 

\mgraphics{graphics/2023-12-13-18-46-34.png}{Switch forwarding table}

\begin{itemize}
    \item Each time a fram is received on interface I, the switch looks at both the source and destination MAC addresses - if the source address appears in the table, update the interface the time, else add an entry with the MAC address, interface and time 
    \item That was the "learning" phase. Now we need to forward the frame - if the destination MAC address doesn't appear in the table, we forward to every interface except the one on which we received the frame. 
    \item If the destination MAC address has an entry in the table and its associated interface is I, discard the frame, else send it to I'
    \item If the destination address is not in the forwarding table, broadcast (except to the interface it just came on)
    \item In addition, we never put the broadcast address - $FF-FF-FF-FF-FF-FF$ in the table and delete entries whose time is too long ago ("too long ago" is defined by the admin)
    \item Again, we need to also check the CRC to ensure the frame has not been corrupted. Notice that in reality, interfaces are fixed and all that really needs to be updated is the timestamp.
\end{itemize}



Now how do you get the destination host's MAC address? Notice that a node knows it own MAC address, and can query for the destination's IP using DNS. So, given the desitination IP, we can the Address Resolution Protocol to get the MAC. There are two cases to consider:
\begin{enumerate}
    \item The nodes are on the same LAN: in which case and ARP query is broadcast and MAC addresses are exchanged 
    \item The nodes are on different subnets: the sender can check the destinations IPs prefix to determine if its on the same subnet or not. If it is not, it just forwards the frame to its default gateway router (notice it knows the default gateways IP via DHCP and its MAC via ARP). The router then creates a new frame forwarding it to the next router and so on until it reaches the destination subnet and eventually the destination node.
\end{enumerate}

\subsection{Physical and link layer issues}
@TODO

\section{Security}
The principles of security in networking:
\begin{itemize}
    \item Confidentiality: only relevant parties should be able to understand the message 
    \item Authentication: parties must know if the data is being sent by a trust source 
    \item Message integrity: data should not be altered 
    \item Access and availability: services must be accessible and available to users 
\end{itemize}

\mgraphics{graphics/2023-12-13-20-34-49.png}{Network secuirty terminology; Alice, Bob and Trudy}

What can Trudy do?
\begin{itemize}
    \item Eavesdrop
    \item Alter messages
    \item Impersonate 
    \item Hijack 
    \item DoS 
\end{itemize}

In comes encryption. How can Trudy break an encryption scheme?
\begin{itemize}
    \item Cipher-text only attack: Trudy only has access to the cipher text and either brute forces or does a statistical analysis
    \item Known-plaintext attack: Trudy has some cipher-text with its plaintext and can try to extrapolate from their 
    \item Chosen-plaintext attack: Trudy can encrypt any plaintext, i.e has access to encryption key and can try to work her way backward
\end{itemize}




\end{document}
